% Also note that the "draftcls" or "draftclsnofoot", not "draft", option
% should be used if it is desired that the figures are to be displayed in
% draft mode.
%
\documentclass[conference]{IEEEtran}
%
\ifCLASSINFOpdf
\usepackage[pdftex]{graphicx}
  % \DeclareGraphicsExtensions{.pdf,.jpeg,.png}
\else
  % \DeclareGraphicsExtensions{.eps}
\fi

% *** MATH PACKAGES ***
%
\usepackage[cmex10]{amsmath}

%\usepackage[caption=false]{caption}
\usepackage[font=footnotesize,caption=false]{subfig}

% --------------- USEPACKAGE agregados por guanucoluis ----------------

\usepackage[utf8]{inputenc}
\usepackage{multirow}
%\usepackage[english]{babel}
\usepackage{amssymb}
%\usepackage[pdftex]{graphicx}
\usepackage[hyphenbreaks]{breakurl}
\usepackage[hyphens]{url}

% ------------------------- Agregados por maxi ------------------------

\renewcommand{\abstractname}{Resumen}
\renewcommand{\figurename}{Fig.}
\renewcommand{\tablename}{Tabla}
\renewcommand{\refname}{Referencias}
\hyphenation{de-sa-rro-llar de-sa-rro-llos de-sa-rro-llo clas-si-fi-can ne-ce-sa-ria-men-te dis-po-si-ti-vos in-te-gra-das es-pa-cio pre-sen-tan di-men-sio-nes di-fe-ren-tes in-dus-tri-al prin-ci-pa-les per-mi-ten com-pu-ta-do-ras pro-por-cio-na dis-po-si-ti-vo im-ple-men-tar par-ti-ci-pa-do di-gi-ta-les rui-do-sa he-rra-mien-tas}

% correct bad hyphenation here
\hyphenation{op-tical net-works semi-conduc-tor}


\begin{document}
%
% paper title
% can use linebreaks \\ within to get better formatting as desired
\title{Revisión -- Bubble Sort: An Archaeological Algorithmic Analysis}


% author names and affiliations
% use a multiple column layout for up to three different
% affiliations
\author{\IEEEauthorblockN{Luis Alberto Guanuco}
\IEEEauthorblockA{Algortímos y Patrones de Software\\
Especialidad en Sistemas Embebidos\\
Instituto Universitario Aeronáutico}
}

% make the title area
\maketitle


\begin{abstract}
El presente documento realiza una revisión de la publicación
\emph{Bubble Sort: An Archaeological Algorithmic Analysis}. Se
rescatan los puntos más importantes de las investigaciones realizadas
por \emph{Owen Astrachan}. El eje central de la publicación es la
revisión histórica del ordenamiento burbuja (\emph{bubble
  sort}). Se buscan los orígenes del algoritmos y la injustificable
vigencia del mismo en los ámbitos académicos informáticos.
\end{abstract}

\IEEEpeerreviewmaketitle

\section{Introducción}
\label{sec:intro}

El autor lleva adelante una investigación profunda de los orígenes del
algoritmo \emph{Bubble Sort}. 

\section{Los orígenes del algoritmo}
\label{sec:origen-alg}

El primer registro del algoritmo se da en el año 1956. En aquella
oportunidad no se lo presenta como \emph{Bubble Sort}, se la expone
como \emph{sorting by exchange}. Las publicaciones que le siguieron a
esta primera aparición del algoritmo siguieron con refiriéndose como
sorting by exchange. 
Una primera aparición del nombre \emph{Bubble Sort} se da en el año
1962.  Kenneth Iverson es el matemático que lo denomina con este
nombre al algoritmo de ordenamiento burbuja.
En el año 1963 el algoritmo ingreso en los repositorios de la ACM
(Association for Computing Machinery) donde el nombre designado fue
\emph{Shuttle Sort}. Luego de su publicación se encontraron
definiciones anexas al algoritmo como \emph{not free from
  errors}. Esto es consecuencia de los errores encontrados en las
implementaciones sucesivas. 

\subsection{Código del Bubble Sort}
\label{sec:origen-bubble-sort}

El autor del paper toma como referencia el código presentado en el
libro \emph{The Art of Computer Programming: Sorting and
  Searching}. El programa recorre todo el vector realizando
comparaciones de dos elementos sucesivos.
\begin{verbatim}
void BubbleSort(Vector a, int n)
{
  for(int j=n-1; j > 0; j--)
    for(int k=0; k < j; k++)
      if (a[k+1] < a[k])
        Swap(a,k,k+1);
}
\end{verbatim}

Sobre este código se presentan dos optimizaciones, vistas también en
la exposición de clases:
\begin{itemize}
\item Caso de tener un vector ordenado.
\item Alternar la  dirección del intercambio.
\end{itemize}

\subsection{Otros nombres para el Bubble Sort}
\label{sec:origen-otro-nombre}

Varias publicaciones posteriores a 1962 presentaron algoritmos con la
estructura del Bubble Sort. Entre ellos están el \emph{Push-Down
  Sort}, \emph{Jump-Down}, \emph{Selection Sort}, por nombrar
algunos. En la evolución de los algoritmos se dan a conocer trabajos
que comparan el Bubble Sort con otros algoritmos de ordenamiento.

\subsection{Los orígenes de popularidad}
\label{sec:origen-popu}

En el año 1971 se publican los resultados de una encuesta realizada
por el ACM sobre los algoritmos de ordenamiento existentes. Sobre el
Bubble Sort se concluyó,
\begin{quote}
El Bubble Sort es fácil de recordar y programar. Además requiere poco
tiempo para completar un simple recorrido.   
\end{quote}

Se relevaron más bibliografías y se encontraron trabajos que denotan
las limitantes del Bubble Sort. Estos se fundamentan en el orden de
complejidad del algoritmo ($O(n^2)$). 

\section{Características de funcionamiento}
\label{sec:car-func}

\section{Conclusiones}
\label{sec:conc}


\begin{thebibliography}{1}
\bibitem{Astrachan}
  Owen~Astrachan, \emph{Bubble Sort: An Archaeological Algorithmic
    Analysis}. Computer Science Departament, Duke University. 
\end{thebibliography}

% that's all folks
\end{document}


